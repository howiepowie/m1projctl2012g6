\begin{figure}[H]
\centering
\begin{subfigure}[b]{.4\textwidth}
\centering
\figscale{
  \begin{tikzpicture}[%
    >=stealth,
    shorten >=1pt,
    node distance=2cm,
    on grid,
    auto,
    state/.append style={minimum size=2em},
    thick,
    scale=0.3
  ]
    \node[state] (N1) {$A$};
    \node[state] (N2) [below left of=N1] {$B$};
    \node[state] (N3) [below right of=N1] {$C$};

    \path[->]
              (N1) edge node {} (N2)
              (N2) edge node {} (N3);
  \end{tikzpicture}
}
\caption{Graphe d'états}
\end{subfigure}
\begin{subfigure}[b]{.4\textwidth}
\centering
\begin{minipage}{0.68\linewidth}
\begin{verbatim}
EX(($B && EX($C))) = { 0 }
  ($B && EX($C)) = { 1 }
    $B = { 1 }
    EX($C) = { 1 }
      $C = { 2 }
\end{verbatim}
\end{minipage}
\caption{Preuve au format textuel}
\end{subfigure}
\caption{Exemple de preuve pour la formule $EX(\$B~\&\&~EX(\$C))$ sur l'état 0}
\label{fig:PreuveAffichageTextuel}
\end{figure}