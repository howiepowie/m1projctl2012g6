\documentclass[dvipsnames]{report}
\usepackage[T1]{fontenc}
\usepackage[utf8]{inputenc}
\usepackage[francais]{babel}
\usepackage[top=2cm, bottom=2cm, left=2cm, right=2cm]{geometry}
\usepackage{array}
\usepackage{amsmath}
\usepackage{pgfplots}
\usepackage{tikz}
\usepackage{caption}
\usepackage{subcaption}
\usepackage{listings}
\usepackage{color}
\usepackage{float}
\usepackage{tabularx}
\usetikzlibrary{automata,positioning}
\setcounter{secnumdepth}{0}

\lstset{% This applies to ALL lstlisting
    backgroundcolor=\color{white!10},%
  language=Java,                % the language of the code
  basicstyle=\footnotesize,           % the size of the fonts that are used for the code
  showspaces=false,               % show spaces adding particular underscores
  showstringspaces=false,         % underline spaces within strings
  showtabs=false,                 % show tabs within strings adding particular underscores
  frame=single,                   % adds a frame around the code
  rulecolor=\color{black},        % if not set, the frame-color may be changed on line-breaks within not-black text (e.g. comments (green here))
  tabsize=2,                      % sets default tabsize to 2 spaces
  keywordstyle=\color{blue},          % keyword style
  commentstyle=\color{gray},
  captionpos=b,
  literate={á}{{\'a}}1 {ã}{{\~a}}1 {é}{{\'e}}1 {è}{{\`e}}1 {à}{{\`a}}1,
    }%

\protected\def\psverb#1{\def\innerpsverb##1#1{\texttt{##1}}\innerpsverb}

\def\fixcheck{%
    \def\tikzifinpicture##1##2{##2}%
}%

% type user-defined commands here

\begin{document}

\title{Modélisation et vérification}   % type title between braces
\author{Hamza Benkabbou, Kevin Bourgeois, Jérémy Morosi,\\Jean-Baptiste Perrin, Zo Rabarijaona}         % type author(s) between braces
\date{\today}    % type date between braces
\maketitle

\newpage

\section{Justification}

Cette partie décrit la manière dont le programme calcule et mémorise la justification d'une formule CTL pour ensuite en donner une représentation visuelle.

\subsection{Preuve}

L'interface \texttt{IPreuve} (listing \ref{lst:IPreuve}) définie les méthodes nécessaires pour prouver qu'une formule CTL est vraie. 

La structure d'une preuve est très simple, puisqu'elle consiste en la formule qui lui est associée (\texttt{getFormule}), en la liste des états qui sont vrais (\texttt{getMarquage}, \texttt{setMarquage}) et en la liste des sous-preuves pour chaque morceau de la formule (\texttt{getPreuves}).

Une preuve dispose aussi de plusieurs méthodes pour la visualiser. On peut récupérer la formule qui lui est associée au format textuel grâce à la méthode \texttt{formuleToString}. On peut obtenir un affichage sous forme d'arbre de la preuve et de ses sous-preuves grâce à la méthode \texttt{toTree}. On peut récupérer la formule sous la forme d'un label coloré pour l'exportation au format \texttt{.dot} grâce à la méthode \texttt{toDotLabel}. Et finalement, on peut exporter la preuve au format \texttt{.dot} grâce aux méthodes \texttt{toDotRacine} et \texttt{toDot}.

Lors du calcul de la preuve, les méthodes \texttt{couperRacine} et \texttt{couper} permettent de supprimer les états qui sont inutiles car ils n'ont pas de prédécesseurs dans la preuve parent.

\subsubsection{Création}

La création d'une preuve se fait en même temps que la validation de la formule CTL associée. Si la formule est un ensemble de sous-formules, on commence d'abord par les justifier. Ensuite, on peut créer une preuve du même type que la formule, en lui donnant le marquage la validant et la liste des sous-preuves. Au passage, on génère la couleur et le label (au format \texttt{.dot}) de la formule. Le label est l'ensemble des labels des sous-formules auxquels on ajoute le nom de la formule avec la couleur correspondante. Par exemple, pour $E(\$A \cup \$B)$, on obtient le label \texttt{<FONT COLOR="...">E(... U ...)</FONT>}.

\subsubsection{Coupage}

Une fois qu'on a construit la preuve complète, on appelle la méthode \texttt{couperRacine} en donnant en paramètre le numéro de l'état pour lequel on doit justifier la formule CTL. Cet appel a pour conséquence de mettre tous les états validant la preuve à $false$, sauf celui donné. La preuve appelle ensuite la méthode \texttt{couper} des sous-preuves en donnant son marquage en paramètre. Les sous-preuves vont alors remettre à $false$ tous les états valident qui n'ont pas de prédécesseurs dans le marquage donné, puis, elles vont à leur tour appeler la méthode pour leurs sous-preuves en donnant leur propre marquage. Ainsi, à la fin, il ne restera dans les preuves que les états qui sont réellement nécessaires à la justification.

\subsubsection{Affichage au format textuel}

\subsubsection{Affichage au format .dot}

L'exportation de la preuve au format \texttt{.dot} commence dans la méthode \texttt{justifieToDot} de la classe \texttt{GrapheRdP}. On appelle la méthode \texttt{toDotRacine} de la preuve complète en lui donnant en paramètres une variable de type \texttt{Map<Integer, {Set<Integer>}>} et une variable de type \texttt{Set<String>}. La première, dont les clefs sont les états du réseau de pétri et les valeurs sont les listes des états auxquels ils sont reliés, permettra à toutes les sous-preuves d'indiquer quels états sont reliés par une justification. La seconde devra contenir l'ensemble des flèches du graphe (au format \texttt{.dot}) qui font parties de la justification.

La méthode \texttt{toDotRacine} sert uniquement à appeler la méthode \texttt{toDot} pour toutes les sous-preuves sans qu'une flèche ne soit ajoutée pour l'état de départ. La méthode \texttt{toDot} permet à une preuve d'ajouter toutes les flèches nécessaires au graphe puis de s'appeler pour toutes les sous-preuves. Une flèche doit partir d'un état parent, aller vers un état validant la sous-preuve et elle doit avoir le label de la formule associée à la sous-preuve à côté.

Une fois que toutes les flèches ont été ajoutées, on commence par exporter la liste des états avec leur marquage et avec le label de la formule complète pour l'état pour lequel on doit justifier la formule (figure \ref{fig:PreuveAffichageDot1}). On exporte ensuite l'ensemble des flèches qui se trouvent dans la variable précédente (figure \ref{fig:PreuveAffichageDot2}). Puis on complète le graphe avec les flèches manquantes (celles qui ne font pas parties de la justification) (figure \ref{fig:PreuveAffichageDot3}).

\begin{figure}[H]
\centering
\begin{subfigure}[b]{.3\textwidth}
\centering
  \begin{tikzpicture}[%
    >=stealth,
    shorten >=1pt,
    node distance=2cm,
    on grid,
    auto,
    state/.append style={minimum size=2em},
    thick,
    scale=0.3
  ]
    \node[state, rectangle, align=center] (N1) {A\\$EX(\$B)$};
    \node[state] (N2) [below right of=N1] {$B$};
    \node[state] (N3) [below left of=N1] {$C$};
  \end{tikzpicture}
\caption{Preuve complète}
\label{fig:PreuveAffichageDot1}
\end{subfigure}
\begin{subfigure}[b]{.3\textwidth}
\centering
  \begin{tikzpicture}[%
    >=stealth,
    shorten >=1pt,
    node distance=2cm,
    on grid,
    auto,
    state/.append style={minimum size=2em},
    thick,
    scale=0.3
  ]
    \node[state, rectangle, align=center] (N1) {A\\$EX(\$B)$};
    \node[state] (N2) [below right of=N1] {$B$};
    \node[state] (N3) [below left of=N1] {$C$};

    \path[->]
              (N1) edge node {$\$B$} (N2);
  \end{tikzpicture}
\caption{Sous-preuves}
\label{fig:PreuveAffichageDot2}
\end{subfigure}
\begin{subfigure}[b]{.3\textwidth}
\centering
  \begin{tikzpicture}[%
    >=stealth,
    shorten >=1pt,
    node distance=2cm,
    on grid,
    auto,
    state/.append style={minimum size=2em},
    thick,
    scale=0.3
  ]
    \node[state, rectangle, align=center] (N1) {A\\$EX(\$B)$};
    \node[state] (N2) [below right of=N1] {$B$};
    \node[state] (N3) [below left of=N1] {$C$};

    \path[->]
              (N1) edge node {$\$B$} (N2)
              (N1) edge node {} (N3);
  \end{tikzpicture}
\caption{Flèches manquantes}
\label{fig:PreuveAffichageDot3}
\end{subfigure}
\caption{Affichage de la preuve pour $EX(\$B)$}
\end{figure}

Pour l'ajout des flèches manquantes, on utilise la liste des états reliés pour savoir si elle existe déjà. La liste des flèches nous permet de nous assurer qu'une même preuve ne soit affichée qu'une fois (figure \ref{fig:PreuveAffichageDot4}).

\begin{figure}[H]
\centering
  \begin{tikzpicture}[%
    >=stealth,
    shorten >=1pt,
    node distance=2cm,
    on grid,
    auto,
    state/.append style={minimum size=2em},
    thick,
    scale=0.4
  ]
    \node[state, rectangle, align=center] (N1) {$A$\\$AX(EX(EX(\$E)))$};
    \node[state] (N2) at (-4, -4) {$B$};
    \node[state] (N3) at (4, -4) {$C$};
    \node[state] (N4) at (0, -8) {$D$};
    \node[state] (N5) at(4, -8) {$E$};

\path[->]
	(N1) edge node [above left, pos=0.75] {$EX(EX(\$E))$} (N2)
	(N1) edge node [above right, pos=0.75] {$EX(EX(\$E))$} (N3)
	(N2) edge node [below left, pos=0.25] {$EX(\$E)$} (N4)
	(N3) edge node [below right, pos=0.25] {$EX(\$E)$} (N4)
	(N4) edge node [below] {$\$E$} (N5);
  \end{tikzpicture}
\caption{Affichage de la preuve pour $AX(EX(EX(\$E)))$}
\label{fig:PreuveAffichageDot4}
\end{figure}

\subsubsection{Coloration}

La coloration des preuves est gérée par la classe \texttt{Coloration} (listing \ref{lst:Coloration}). Son but est d'associer une couleur unique (si possible) et un label coloré (pour l'exportation au format \texttt{.dot}) à une formule. Lors de l'exportation, les accesseurs \texttt{getCouleur} et \texttt{getLabel} permettront alors de récupérer les informations liées à la formule donnée.

Les couleurs sont générées par la méthode \texttt{genererCouleur}. Dans l'idéal, toutes les couleurs utilisées doivent être uniques et ne doivent pas trop se ressembler pour que la preuve soit suffisamment claire. Ici, on génère juste des couleurs au format HSV avec $s=1$, $v=0.75$ et en incrémentant le $h$ d'une certaine valeur à chaque appel de la méthode.

La couleur d'une formule est utilisée pour colorer une flèche partant d'un état validant la formule et allant vers un état validant une sous-formule alors que le label de la sous-formule est affiché sur la flèche (figure \ref{fig:Coloration}).

\begin{figure}[ht]
\centering
\begin{subfigure}[b]{.4\textwidth}
\centering
  \begin{tikzpicture}[%
    >=stealth,
    shorten >=1pt,
    node distance=2cm,
    on grid,
    auto,
    state/.append style={minimum size=2em},
    thick,
    scale=0.4
  ]
    \node[state, rectangle, align=center] (N1) {$A$\\$\textcolor{BrickRed}{EX(\textcolor{Blue}{\$B})}$};
    \node[state] (N2) [below right of=N1] {$B$};
    \node[state] (N3) [below left of=N1] {$C$};

    \path[->, BrickRed]
              (N1) edge node {$\textcolor{Blue}{\$B}$} (N2);

    \path[->]
              (N1) edge node {} (N3);
  \end{tikzpicture}
\caption{$EX(\$B)$}
\end{subfigure}
\begin{subfigure}[b]{.4\textwidth}
\centering
  \begin{tikzpicture}[%
    >=stealth,
    shorten >=1pt,
    node distance=2cm,
    on grid,
    auto,
    state/.append style={minimum size=2em},
    thick,
    scale=0.4
  ]
    \node[state, rectangle, align=center] (N1) {$A$\\$\textcolor{BrickRed}{AX(\textcolor{Cerulean}{EX(\textcolor{RoyalPurple}{EX(\textcolor{OliveGreen}{\$E})})})}$};
    \node[state] (N2) at (4, -4) {$B$};
    \node[state] (N3) at (-4, -4) {$C$};
    \node[state] (N4) at (0, -8) {$D$};
    \node[state] (N5) at(4, -8) {$E$};

\path[->, BrickRed]
	(N1) edge node [above right, pos=0.75] {$\textcolor{Cerulean}{EX(\textcolor{RoyalPurple}{EX(\textcolor{OliveGreen}{\$E})})}$} (N2)
	(N1) edge node [above left, pos=0.75] {$\textcolor{Cerulean}{EX(\textcolor{RoyalPurple}{EX(\textcolor{OliveGreen}{\$E})})}$} (N3);

\path[->, Cerulean]
	(N2) edge node [below right, pos=0.25] {$\textcolor{RoyalPurple}{EX(\textcolor{OliveGreen}{\$E})}$} (N4)
	(N3) edge node [below left, pos=0.25] {$\textcolor{RoyalPurple}{EX(\textcolor{OliveGreen}{\$E})}$} (N4);

\path[->, RoyalPurple]
	(N4) edge node [below] {$\textcolor{OliveGreen}{\$E}$} (N5);
  \end{tikzpicture}
\caption{$AX(EX(EX(\$E)))$}
\end{subfigure}
\caption{Exemples de coloration}
\label{fig:Coloration}
\end{figure}

\subsection{Preuves atomiques}

Les preuves atomiques sont les preuves associées aux formules atomiques $p$, $true$, $false$, $dead$ et $initial$. Elles correspondent respectivement aux classes \texttt{Atom}, \texttt{True}, \texttt{False}, \texttt{Dead} et \texttt{Initial}. Comme leur implémentation est assez simple puisqu'elle n'ont pas de sous-preuves, on montrera juste les affichages obtenus au format textuel (figure \ref{fig:PreuveAtomiqueTextuel}) et au format \texttt{.dot} (figure \ref{fig:PreuveAtomiqueDot}) (sauf pour $false$ qui ne peut pas être affichée).

\begin{figure}[ht]
\centering
\begin{subfigure}[b]{.24\textwidth}
\centering
  \begin{tikzpicture}[%
    >=stealth,
    shorten >=1pt,
    node distance=2cm,
    on grid,
    auto,
    state/.append style={minimum size=2em},
    thick,
    scale=0.2
  ]
    \node[state, rectangle, align=center] (N1) {$A$\\$\textcolor{BrickRed}{EX(\textcolor{Blue}{\$B})}$};
    \node[state] (N2) [below left of=N1] {$B$};
    \node[state] (N3) [below right of=N1] {$C$};

\path[->, BrickRed]
	(N1) edge node [above left] {$\textcolor{Blue}{\$B}$} (N2);

\path[->]
	(N1) edge [bend right] node {} (N3)
	(N3) edge node {} (N1);
  \end{tikzpicture}
\caption{$EX(\$B)$}
\end{subfigure}
\begin{subfigure}[b]{.24\textwidth}
\centering
  \begin{tikzpicture}[%
    >=stealth,
    shorten >=1pt,
    node distance=2cm,
    on grid,
    auto,
    state/.append style={minimum size=2em},
    thick,
    scale=0.2
  ]
    \node[state, rectangle, align=center] (N1) {$A$\\$\textcolor{BrickRed}{EX(\textcolor{Blue}{true})}$};
    \node[state] (N2) [below left of=N1] {$B$};
    \node[state] (N3) [below right of=N1] {$C$};

\path[->, BrickRed]
	(N1) edge node [above left] {$\textcolor{Blue}{true}$} (N2);

\path[->]
	(N1) edge [bend right] node {} (N3)
	(N3) edge node {} (N1);
  \end{tikzpicture}
\caption{$EX(true)$}
\end{subfigure}
\begin{subfigure}[b]{.24\textwidth}
\centering
  \begin{tikzpicture}[%
    >=stealth,
    shorten >=1pt,
    node distance=2cm,
    on grid,
    auto,
    state/.append style={minimum size=2em},
    thick,
    scale=0.2
  ]
    \node[state, rectangle, align=center] (N1) {$A$\\$\textcolor{BrickRed}{EX(\textcolor{Blue}{dead})}$};
    \node[state] (N2) [below left of=N1] {$B$};
    \node[state] (N3) [below right of=N1] {$C$};

\path[->, BrickRed]
	(N1) edge node [above left] {$\textcolor{Blue}{dead}$} (N2);

\path[->]
	(N1) edge [bend right] node {} (N3)
	(N3) edge node {} (N1);
  \end{tikzpicture}
\caption{$EX(dead)$}
\end{subfigure}
\begin{subfigure}[b]{.24\textwidth}
\centering
  \begin{tikzpicture}[%
    >=stealth,
    shorten >=1pt,
    node distance=2cm,
    on grid,
    auto,
    state/.append style={minimum size=2em},
    thick,
    scale=0.2
  ]
    \node[state] (N1) {$A$};
    \node[state] (N2) [below left of=N1] {$B$};
    \node[state, rectangle, align=center] (N3) [below right of=N1] {$C$\\$\textcolor{BrickRed}{EX(\textcolor{Blue}{initial})}$};

\path[->, BrickRed]
	(N3) edge node [above right] {$\textcolor{Blue}{initial}$} (N1);

\path[->]
	(N1) edge node {} (N2)
	(N1) edge [bend right] node {} (N3);
  \end{tikzpicture}
\caption{$EX(initial)$}
\end{subfigure}
\caption{Affichages au format \texttt{.dot} obtenus pour les preuves atomiques}
\label{fig:PreuveAtomiqueDot}
\end{figure}

\begin{figure}{ht}

\end{figure}

\section{Utilisation du programme}

\subsection{Invité de commande}

Par défaut, le programme s'utilise en invité de commande. Dans ce mode, il affiche le symbole \texttt{>} et attend la saisie d'une commande. Les commandes données sont parsées à l'aide d'un parseur généré par ANTLR (voir \texttt{principal/CommandLine.g}) qui va se charger d'appeler les méthodes correspondantes dans la classe \texttt{Main}.

Les commandes disponibles sont toutes celles demandées dans le sujet ainsi que quelques unes qui ont été ajoutées.

\subsection{Commandes}

\noindent\begin{tabularx}{\linewidth}{@{}l l X@{}}
load & <fichier.net> & charge le réseau de pétri depuis le fichier \\
graphe & & calcule le graphe des états accessibles \\
look & <etat> & affiche le marquage de l'état \\
succ & <etat> & affiche la liste des successeurs de l'état \\
todot & <fichier.dot> & exporte le graphe au format \texttt{.dot} dans le fichier \\
ctl & <formule> & affiche le nombre d'états validant la formule \\
ctl & <formule> <etat> & affiche $vrai$ si l'état valide la formule ou $faux$ \\
ctltodot & <formule> <fichier.dot> & exporte le graphe au format \texttt{.dot} en colorant les états qui valident la formule \\
Justifie & <formule> <etat> & affiche la preuve au format textuel que l'état valide la formule \\
Justifietodot & <formule> <etat> <fichier.dot> & exporte le graphe et la preuve que l'état valide la formule au format \texttt{.dot} dans le fichier \\
shell & & passe en mode shell \\
stop & & arrête le programme \\
\end{tabularx}

\subsection{Création d'un script}

Outre le mode invité de commande, il est possible de créer un script pour lancer le programme et lui faire exécuter un ensemble de commandes. Les différents exemples fournis avec le projet fonctionnent ainsi:\\

On lance le programme normalement avec \texttt{java -jar modelprojet.jar} et on lui donne une liste de paramètres entre les deux balises \texttt{END\_PARAMS}.
\begin{verbatim}
java -jar modelprojet.jar << -END_PARAMS
    ...
END_PARAMS
\end{verbatim}

On commence par exécuter la commande \texttt{shell} qui permet au programme de passer en mode shell. Dans ce mode, toutes les commandes exécutées sont affichées après le symbole \texttt{>}. Ce mode est nécessaire car les commandes fournies au programme entre les balises \texttt{END\_PARAMS} ne sont pas affichées dans la console comme si elles avaient été saisies par l'utilisateur. La dernière commande à exécuter doit être la commande \texttt{stop} pour demander au programme de s'arrêter une fois le script fini.

\begin{verbatim}
java -jar modelprojet.jar << -END_PARAMS
    shell
    ...
    stop
END_PARAMS
\end{verbatim}

C'est entre les deux commandes \texttt{shell} et \texttt{stop} que l'on peut ajouter toutes les commandes du script.

\begin{verbatim}
java -jar modelprojet.jar << -END_PARAMS
    shell
    load "hello.net"
    graphe; todot "hello.dot"
    ctl EX(EX(\$C)); ctl EX(\$C)
    stop
END_PARAMS
\end{verbatim}

\^{A} noter que des commandes séparées par un retour à la ligne produiront le même effet que si l'utilisateur avait saisi la première, appuyé sur entrée puis avait saisi la seconde. \^{A} l'inverse, deux commandes séparées par une point-virgule seront parsées et exécutées à la suite comme si elles avaient été saisies en même temps par l'utilisateur.

La différence est que, dans le second cas, les résultats des deux commandes seront affichés à la suite sans savoir quelle commande a produit quel résultat. Alors que dans le premier cas, le résultat de la première commande sera séparé du résultat de la seconde par le symbole \texttt{>} et l'affichage de la seconde commande.

\section{Exemples}

De nombreux exemples ont été créés et sont disponibles dans le dossier d'exemples du projet. Ces exemples sont sous la forme d'un script \texttt{.sh} qui doit se trouver dans le même dossier que le \texttt{.jar} du programme.\\

Chaque exemple contient des commentaires indiquant son intérêt et les fichiers produits. Cependant, voici une liste récapitulative des plus importants:\\

\noindent\begin{tabularx}{\linewidth}{| @{}l | X@{} |}
\hline
Script \texttt{.sh} & Description \\
\hline
test commandes & exécute toutes les commandes demandées dans le sujet sur le fichier \texttt{hello.net} \\
\hline
test justifie & produit une justification au format textuel et \texttt{.dot} de l'ensemble des formules possibles sur les différents fichiers \texttt{.net} \\
\hline
preuves atomiques & produit une justification au format textuel et \texttt{.dot} des formules atomiques $p$, $true$, $dead$ et $initial$ sur le fichier \texttt{atomique.net} \\
\hline
\end{tabularx}

\newpage

\begin{lstlisting}[caption={Interface \psverb+IPreuve+ commune à toutes les preuves}, label={lst:IPreuve}]
public interface IPreuve {

	public Tree getFormule();
	public boolean[] getMarquage();
	public void setMarquage(boolean[] marquage);
	public List<IPreuve> getPreuves();

	public void couperRacine(CTL ctl, int[][] pred, int etat);
	public void couper(CTL ctl, int[][] pred, boolean[] parents);

	public String toTree();
	public String toTree(String indent);

	public void toDotRacine(Map<Integer, Set<Integer>> fleches,
			Set<String> justifications, IPreuve parent, int etatParent,
			Coloration couleurs);
	public void toDot(Map<Integer, Set<Integer>> fleches,
			Set<String> justifications, IPreuve parent, int etatParent,
			Coloration couleurs);

	public String toDotLabel(Coloration couleurs);
	public IPreuve clone();
	public String formuleToString();

}
\end{lstlisting}

\begin{lstlisting}[caption={Classe \psverb+Coloration+}, label={lst:Coloration}]
public class Coloration {

	private Map<Tree, String> couleursFormules;
	private Map<Tree, String> labelsFormules;

	public String getCouleur(Tree formule);
	public String getLabel(Tree formule);

	public void ajouter(Tree formule, String couleur, String label);
	public FakeTree ajouter(String label);

	public String genererCouleur();

}
\end{lstlisting}

\end{document}