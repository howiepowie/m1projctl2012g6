\begin{figure}[H]
\centering
\begin{subfigure}[b]{.4\textwidth}
\centering
\figscale{
\begin{tikzpicture}[%
    >=stealth,
    shorten >=1pt,
    node distance=2cm,
    on grid,
    auto,
    state/.append style={minimum size=2em},
    thick,
    scale=0.4
  ]
    \node[state, rectangle, align=center] (N1) {$A$\\Contre exemple\\$\textcolor{BrickRed}{AX(\textcolor{ForestGreen}{!\textcolor{Blue}{\$D}})}$};
    \node[state] (N2) at (-5, -5) {$B$};
    \node[state] (N3) at (5, -5) {$C$};
\node[state] (N4) at (0, -10) {$D$};

    \path[->, BrickRed]
              (N1) edge node [above left] {$\textcolor{ForestGreen}{!\textcolor{Blue}{\$D}}$} (N2)
              (N1) edge node [above right] {$\textcolor{ForestGreen}{!\textcolor{Blue}{\$D}}$} (N3);

    \path[->]
              (N2) edge node {} (N4)
(N3) edge node {} (N4);
  \end{tikzpicture}
}
\caption{Affichage \texttt{.dot}}
\label{fig:PreuveContreExempleDot}
\end{subfigure}
\begin{subfigure}[b]{.4\textwidth}
\centering
\begin{minipage}{1\linewidth}
\begin{verbatim}
> Justifie EX($D) 0
L'état donné ne valide pas la formule.
Justification de AX(!$D) pour l'état 0.
AX(!$D) = { 0 }
  !$D = { 1 }
    $D = { 2 }
  !$D = { 3 }
    $D = { 2 }
\end{verbatim}
\end{minipage}
\caption{Affichage textuel}
\label{fig:PreuveContreExempleTextuel}
\end{subfigure}
\caption{Contre exemple de $EX(\$D)$ pour l'état 0}
\label{fig:PreuveContreExemple}
\end{figure}